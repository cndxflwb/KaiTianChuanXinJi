% xelatex
\documentclass[UTF8,scheme=chinese]{ctexbook}

\ctexset{
	%punct=kaiming,%開明式標點
	chapter/numbering=false,                  %章級標題不編號
	chapter/format=\Large\centering\bfseries, %居中標題
	chapter/pagestyle=empty,                  %章級標題頁面不顯示頁眉頁腳
	section/numbering=false,                  %節級標題不編號
	section/format=\large\bfseries, %居中標題
	contentsname={目\quad 録},                %重命名目錄名稱
}

% 尾注-用於排版校勘記
% 用法:文中\endnote{...} 在要显示尾注的位置输入 \theendnotes
\usepackage{endnotes}
\usepackage{etoolbox}% 每章尾注重新编号
\makeatletter
\def\enoteheading{\subsubsection*{\notesname
  \@mkboth{\MakeUppercase{\notesname}}{\MakeUppercase{\notesname}}}%
  \mbox{}\par\vskip-3\baselineskip\noindent\rule{.5\textwidth}{0pt}\par\vskip\baselineskip}
\makeatother
\renewcommand{\notesname}{\raggedright 校勘記} %定义尾注的标题名称
\def\enotesize{\normalsize}
\csappto{theendnotes}{\setcounter{endnote}{0}}

\usepackage[paperwidth=442.26pt,paperheight=589.26pt,text={220.5pt,442.26pt},bottom=63.63pt,outer=168pt,marginparwidth=126pt]{geometry}%30字27行,字號10.5bp,左右下邊空6字(63.63bp)
%\usepackage[paperwidth=442.26pt,paperheight=589.26pt,text={315pt,442.26pt},bottom=63.63pt,right=63.63pt]{geometry}

\raggedbottom

\usepackage{xeCJKfntef}

\usepackage{zhnumber}
\usepackage{xcolor}

\usepackage{xeCJK}
\setCJKmainfont[BoldFont={Source Han Serif CN SemiBold}]{Source Han Serif CN}
\xeCJKsetup{
	AutoFallBack=true,
	CheckSingle=false}
\setCJKfallbackfamilyfont{\CJKrmdefault}{{ZhongHuaSongPlane00},{ZhongHuaSongPlane02},{ZhongHuaSongPlane15}}


\usepackage{snotez}
\setsidenotes{
	note-mark-format=〔\zhnumber{#1}〕,%注释文本
	text-mark-format=\raisebox{4pt}{\footnotesize〔\zhnumber{#1}〕},%文中的注释序号
	text-format+=\kaishu\raggedright,%注释文本的格式
	perpage=true%每页重新编号,需要多次编译
}

\newCJKfontfamily\utffont{FZKaiS(SIP)-PUA.TTF}% 用法{\utffont{...}}

\usepackage{fontspec}
\newcommand\puafont{\fontspec[ExternalLocation=]{FZKaiS(SIP)-PUA.TTF}}% 用法{\puafont ...}

\newcommand{\pagenum}{\footnotesize\heiti\color{red}}
\newcommand{\yuanshuzhu}{\small\kaishu}

\usepackage{emptypage}%設置空白页页眉页脚不顯示任何內容

% 目录樣式
\usepackage{titletoc}
%\usepackage[toc]{multitoc}
%\titlecontents{chapter}[0em]{\vspace{3mm}\bfseries}{}%
%{}{\titlerule*[0.5pc]{$\cdot$}\small\contentspage}
\titlecontents{chapter}[0bp]{\addvspace{2pt}\bfseries\filright}%顶格
	{\contentspush{\thecontentslabel\ }}
	{}{\titlerule*[6pt]{$\cdot$}\contentspage}
\titlecontents{section}[\ccwd]{\addvspace{2pt}\filright}%缩进1个汉字
	{\contentspush{\thecontentslabel\ }}
	{}{\titlerule*[6pt]{$\cdot$}\contentspage}
\titlecontents{subsection}[2\ccwd]{\addvspace{2pt}\filright}%缩进2个汉字
	{\contentspush{\thecontentslabel\ }}
	{}{\titlerule*[6pt]{$\cdot$}\contentspage}
% 頁碼對齊
\makeatletter
\renewcommand{\@pnumwidth}{1.5\ccwd}% 预估最大页码要占多宽的位置,正文之前使用的是中文汉字页码,预估最大不超过99,因此设置为2个汉字宽度
\makeatother

\begin{document}

\newgeometry{
	%paperwidth=442.26pt,
	%paperheight=589.26pt,
	textwidth=315pt,
	textheight=442.26pt,
	bottom=63.63pt,
	right=63.63pt,
}%

\chapter{點校説明}

《開天傳信記》一卷,唐鄭綮撰。本書記載開元、天寶間遺聞佚事三十餘條,故名。晁公武《郡齋讀書志》題作《開元傳信記》,失作者著書之旨,顯誤。

鄭綮,字蘊武,排行五,滎陽(今屬河南)人。登進士第,歷監察御史、倉部員外郎、户部員外郎、吏部員外郎。《開天傳信記》,即作于吏部員外郎任上。後又任金、刑、右司郎中,出爲廬州刺史。王徽爲御史中丞,奏薦綮爲兵部郎中、知臺雜,遷給事中,轉右散騎常侍。昭宗於乾寧元年擢其爲禮部侍郎、同中書門下平章事,綮自知難符衆望,自求引退,以太子少保致仕。光化二年(八九九)卒。綮善詩,自云「詩思在灞橋風雪中驢背上」,詩多刺時侮劇,語多俳諧,好用歇後句,時人共號「鄭五歇後體」。今傳詩僅四首,生平事迹見兩《唐書》本傳。

《開天傳信記》被《新唐書·藝文志》、《崇文總目》、晁公武《郡齋讀書志》、尤袤《遂初堂書目》、陳振孫《直齋書録解題》諸官私書目著録如「雜史」類,所記内容爲開元、天寶時事,舉凡朝廷政事、宫掖秘事、社會風情、世態習尚、文學音樂等,無不載述,或可補史載之缺失,如「開元初上{\pagenum -73-}勵精理道」條,叙述開元盛世之繁榮景象,較正史尤爲概括;「上於諸王友愛特甚」條,叙述唐玄宗與諸王友愛情事,生動真切,後被王讜採入《唐語林》中,劉昫著《舊唐書》,亦以之採入《睿宗諸子傳》裏。鄭綮記事亦有失實處,如「上於藩邸」條,四庫館臣根據《舊唐書·王琚傳》王琚選補主簿時乘機進説除太平公主事,又據韋氏稱制,王琚正亡命江都之事,指出:「所記恐非事實,宜爲《通鑑》之所不取。」此外,如「羅公遠多秘術」、「夢遊月宫」、「道士葉法善」、「上見嶽神」諸條,則語涉怪誕,未可傳信。

此書有多種版本,現以丁卯武進陶氏刊行《百川學海》本作底本,參校以南京圖書館藏八千卷樓藏明覆宋本《百川學海》本(簡稱《八千卷》本)、《唐代叢書》本(簡稱《唐代》本)、《歷代小史》本(簡稱《小史》本)、《説庫》本、《説郛》宛委山堂本(簡稱《説宛》本)、《古今説部叢書》本(簡稱《古今説部》本)。王讜《唐語林》、《太平廣記》引録本書不少條目,今據以校補。《説庫》本録本書,其中有六條:「平賊同日」、「三聖子皆登帝位」、「相有二親」、「三代爲相」、「三拜中書」、「三十二年居相位」,不載於《百川學海》本,其他諸本亦無,所述之事大都非開元、天寶事迹,有悖鄭綮之著書宗旨,故本書不予輯録,特此説明。

\begin{flushright}
點校者 二〇〇八年六月{\pagenum -74-}
\end{flushright}

\newgeometry{
	%paperwidth=442.26pt,
	%paperheight=589.26pt,
	bottom=63.63pt,
	outer=168pt,
	textwidth=220.5pt,
	textheight=442.26pt,
	marginparsep=10.5pt,
	marginparwidth=126pt,
}%

\chapter{開天傳信記序}

余何爲者也?累參臺郎,思勤墳典,用自修勵。竊以國朝故事,莫盛於開元、天寶之際。服膺簡策,管窺王業,參於聞聽,或有闕焉。承平之盛,不可殞墜。\CJKunderdot{輒因簿領之暇},\sidenote{「簿領」原作「步領」,今據《全唐文》卷四〇八改。周中孚《鄭堂讀書記》、《四庫全書總目》引録本序,亦作「簿領」。{\pagenum -75-}}搜求遺逸,傳於必信,名曰《開天傳信記》。斗筲微器,周鼎不節之咎,何已遐乎?好事者觀其志,寬其愚,是其心也。

\newgeometry{
	%paperwidth=442.26pt,
	%paperheight=589.26pt,
	textwidth=315pt,
	textheight=442.26pt,
	bottom=63.63pt,
	right=63.63pt,
}%

\tableofcontents

\newgeometry{
	%paperwidth=442.26pt,
	%paperheight=589.26pt,
	bottom=63.63pt,
	outer=168pt,
	textwidth=220.5pt,
	textheight=442.26pt,
	marginparsep=10.5pt,
	marginparwidth=126pt,
}%

\chapter{開天傳信記}

1 上於藩邸時,每戲遊城南韋、杜之間。因逐狡兔,意樂忘返,\CJKunderdot{與其徒十數人},\sidenote{「人」原作「一」,今據《唐代》本、王讜《唐語林》卷四、《太平廣記》卷四九四「王琚」條改。}\CJKunderdot{饑倦甚},\sidenote{「饑」原作「飲」,今據《唐代》本、王讜《唐語林》卷四、《太平廣記》卷四九四改。}\CJKunderdot{休息於封部大樹下}。\sidenote{「封部」《唐代》本、王讜《唐語林》卷四、《太平廣記》卷四九四作「村中」。{\pagenum -77-}}適有書生延上過其家,家貧,止於村妻、一驢而已。上坐未久,書生殺驢拔蒜備饌,酒肉霶霈。上顧而奇之,及與語,磊落不凡,問其姓名,\CJKunderdot{乃王琚也}。\sidenote{《四庫全書總目》卷一四二《開元傳信記》提要云:「其紀明皇戲游城南,王琚延過其家,謀誅韋氏一條,據《唐書·琚傳》,乃琚選補主簿,過謝太子,乘機進説,以除太平公主,並無先過琚家之事。司馬光作《通鑑》,亦不從是書,惟《新唐書》兼採之。然韋氏稱制時,琚方以王同皎黨亡命江都,安得復卜居韋、杜?綮所記恐非事實,宜爲《通鑑》所不取。」}自是上每遊韋杜間,必過琚家,琚所諮議合意,益親善焉。及韋氏專制,上憂甚,獨密言於琚,曰:「亂則殺之,又何疑也?」上遂納琚之謀,戡定禍難。累拜爲中書侍郎,實預配享焉。

\vspace{1.5em}

2 上於諸王友愛特甚,\CJKunderdot{常思作長枕大被},\sidenote{原無「大被」二字,王讜《唐語林》卷一引本書作「常思作長枕大被」,今據補。按《舊唐書·睿宗諸子傳》:「玄宗嘗製一大被長枕,將與成器等共申友悌之好。」{\pagenum -78-}}與諸王同起卧。諸王有疾,上輒終日不食,終夜不寢,形憂於色。左右或開諭進食,上曰:「弟兄,吾手足也。手足不理,吾身廢矣,何暇更思美食安寢邪?」上於東都起五王宅,於上都製花萼相輝之樓,蓋爲諸王爲會集宴樂之地。上與諸王靡日不會聚,或講經義、論理道,間以毬獵蒲博、賦詩飲食,歡笑戲謔,未常惰怠。近古帝王友愛之道,無與比也。

\vspace{1.5em}

3 開元初,上勵精理道,鏟革訛弊,不六七年,天下大治,\CJKunderdot{河清海晏},\sidenote{「海晏」原作「海宴」,今據《八千卷》本、王讜《唐語林》卷三引文改。}物殷俗阜。安西諸國,悉平爲郡縣。自開遠門西行,亘地萬餘里,入河、隍之賦税,\CJKunderdot{滿右藏,東納河北諸道租庸,充滿左藏}。\sidenote{原作「左右藏庫」,今據王讜《唐語林》卷三引文補。}財物山積,不可勝較。四方豐稔,百姓殷富,管户一千餘萬,米一斗三四文,丁壯之人,不識兵器。路不拾遺,行者不囊糧。\CJKunderdot{奇瑞疊應},\sidenote{「奇瑞」原作「其瑞」,今據王讜《唐語林》卷三引文改。}\CJKunderdot{重譯麕至},\sidenote{「重譯」原作「重驛」,今據《説苑》本、《古今説部》本、王讜《唐語林》卷三引文改。}人情欣欣然,感登岱告成之事。上猶惕厲不已,爲讓者數四焉。是時,劉晏年八歲,獻《東封書》。上覽而奇之,命宰相出題,就中書試驗。張説、\CJKunderdot{源乾曜等咸寵薦}。\sidenote{「咸寵薦」王讜《唐語林》卷三引作「咸相感慰薦」。{\pagenum -79-}}上以晏間生秀妙,引宴於内殿,縱六宫觀看。貴妃坐晏於膝上,親爲晏畫眉總丱髻。宫中人投果遺花者,不可勝數也。尋拜晏祕書省正字。

\vspace{1.5em}

4 開元初,山東大蝗。姚元崇請分遣使捕蝗,埋之。上曰:「蝗,天災也,誠由不德而致焉。卿請捕蝗,\CJKunderdot{得無違天而傷義乎}?」\sidenote{原無「天」字,今據王讜《唐語林》卷三引文補。}元崇進曰:「臣聞《大田》詩曰:『秉畀炎火』者,捕蝗之術也。古人行之於前,陛下用之於後,古人行之,所以安農,\CJKunderdot{陛下行之},\sidenote{《八千卷》本作「陛下用之」。}所以除害。臣聞安農,非傷義也,農安則物豐,除害則人豐樂,興農去害,有國之大事也,幸陛下熟思之。」上喜曰:「事既師古,用可救時,是朕心也。」遂行之。時中外咸以爲不可,上謂左右曰:「吾與賢相討論已定,捕蝗之事,敢議者死!」是歲所司結奏,\CJKunderdot{捕蝗蟲凡百餘萬石},\sidenote{王讜《唐語林》卷三引文作「捕蝗十分去四」。}時無飢饉,天下賴焉。

\vspace{1.5em}

5 上將登封太山,益州進白驟至,潔朗豐潤,權奇偉異,上遂親乘之。柔習安便,\CJKunderdot{不知登降之倦}。\sidenote{「登降之倦」《唐代》本作「登降之勞也」。}告成禮畢,復乘而下。纔下山坳,休息未久,而有司言白騾無疾而殪。上歎異{\pagenum -80-}之,謚曰「白騾將軍」,命有司具槽櫝,疊石爲墓,\CJKunderdot{在封禪壇北一里餘},\sidenote{「一里餘」《唐代》本作「數里」。}于今存焉。

\vspace{1.5em}

6 車駕次華陰,上見嶽神數里迎謁。上問左右,\CJKunderdot{左右莫之見}。\sidenote{原無「左右」二字,今據《太平廣記》卷二八三補。{\pagenum -81-}}遂詔諸巫問神安在,獨老巫阿馬婆奏云:「三郎,在路左,朱髮紫衣,迎候陛下。」上顧笑之,仍勒阿馬婆勅神先歸。上至廟,見神櫜鞬,俯伏庭東南大栢樹下,又召阿馬婆問之,對如上見。上加敬禮,命阿馬婆致意,而旋降詔先諸嶽,封爲金天王,仍上自書製碑文以寵異之。其碑髙五十餘尺,闊丈餘,厚四五尺,天下碑莫比也。其陰刻扈從太子、王公以下百官名氏。製作壯麗,\CJKunderdot{鐫刻精巧},\sidenote{「巧」上三字原無,今據《太平廣記》卷二八三補。}無倫比焉。

\vspace{1.5em}

7 上爲皇孫時,風表瓌異,神彩英邁,嘗於朝堂叱武攸暨曰:「朝堂,我家朝堂,汝得恣蜂蠆而狼顧耶?」則天聞而驚異之,再三顧曰:「\CJKunderdot{此兒氣概不常},\sidenote{原無「不常」二字,今據《説宛》本、《古今説部》本補。}終當爲吾家太平天子也。」

\vspace{1.5em}

8 西凉州俗好音樂,製新曲曰《涼州》,開元中列上獻。上召諸王便殿同觀。曲終,諸王賀,舞蹈稱善,獨寧王不拜。上顧問之,寧王進曰:「此曲雖嘉,臣有聞焉,夫音者,始於宫,散於商,成於角、徵、羽,莫不\CJKunderdot{根柢囊橐於宫、商也}。\sidenote{《唐代》本、《太平廣記》卷二〇四作「根蒂而襲於宫商」。王讜《唐語林》卷三作「根柢橐籥於宫商也」。}斯曲也,\CJKunderdot{宫離而少徵},\sidenote{「宫離」《説苑》本、《古今説部》本、王讜《唐語林》卷三作「宫雜」。}商亂而加暴。臣聞宫,君也;商,臣也。\CJKunderdot{宫不勝則君勢卑},\sidenote{「君」原作「商」,今據王讜《唐語林》卷三、《太平廣記》卷二〇四改。}商有餘則臣事僭,\CJKunderdot{卑則畏下},\sidenote{「畏」原作「逼」,今據王讜《唐語林》卷三改。}僭則犯上。\CJKunderdot{發於忽微},\sidenote{「忽微」《説苑》本、《古今説部》本作「隱微」。}形於音聲,播於歌詠,見之於人事。臣恐一日有播越之禍,\CJKunderdot{悖逼之患},\sidenote{《唐代》本作「悖逆之患」,王讜《唐語林》卷三作「悖亂之患」。}莫不兆於斯曲也。」上聞之默然。及安史作亂,華夏鼎沸,所以見寧王審音之妙也。{\pagenum -82-}

\vspace{1.5em}

9 天寶中,上以三河道險束,漕運艱難,\CJKunderdot{乃旁北山鑿石爲月河},\sidenote{《唐代》本作「乃令旁山鑿石爲月河」。}以避湍急,名曰天寶河。歲省運夫五十萬,久無覆溺淹滯之患,天下稱之。其河東西徑直長五里餘,闊四五丈,深三四丈,皆鑿堅石。匠人於石得古鐵鐷,長三尺餘,上有「平陸」二字,皆篆文也。上異之,藏於内庫。遂命改河北縣爲平陸縣,旌其事也。{\pagenum -83-}

\vspace{1.5em}

10 \CJKunderdot{上御勤政樓大酺},\sidenote{「勤政樓」《資治通鑑》卷二一四玄宗開元二十三年載此事作「五鳳樓」,鐵易《南部新書》甲載此事作「花萼樓」。}縱士庶觀看。百戲競作,\CJKunderdot{人物填咽}。\sidenote{「填咽」《説苑》本、王讜《唐語林》本作「填喧」。}金吾衛士白棒雨下,不能制止。上患之,謂力士曰:「吾以海内豐稔,四方無事,故盛爲宴樂,與百姓同歡,不知下人喧亂如此,汝何方止之?」力士曰:「臣不能也。陛下試召嚴安之處分打場,以臣所見,必有可觀。」上從之。安之到,則周行廣場,以手板畫地示衆曰:「\CJKunderdot{犯此者死}!」\sidenote{王讜《唐語林》卷一、《太平廣記》卷一六四作「踰此者必死」。{\pagenum -84-}}\CJKunderdot{以是終五日酺宴},\sidenote{「終五日」《資治通鑑》作「盡三日」。王讜《唐語林》作「終日」。}咸指其地畫曰「嚴公界境」,無一人敢犯者。

\vspace{1.5em}

11 蘇瓌初未知頲,常處頲於馬廐中,與傭僕雜作。一日,有客詣瓌,候於廳所,頲擁篲趍庭,遺墜文書。客取視之,乃詠崑崙奴詩也。其詞曰:「\CJKunderdot{指頭十挺墨},\sidenote{王讜《唐語林》卷三作「指如十挺墨」。}\CJKunderdot{耳朶兩張匙}。\sidenote{王讜《唐語林》卷三作「耳似兩張匙」。}」客心異之。而瓌出,與客淹留。客笑語之餘,因詠其詩,並言形貌,問:「何人?非足下宗族庶孽耶?」\CJKunderdot{瓌備言其事,客驚賀之,請瓌}加禮收舉,\sidenote{原無此十一字,今據《唐代》本、王讜《唐語林》卷三補。}必蘇氏之令子也。瓌自是稍稍親之。適有人獻瓌兔,懸於廊廡間。瓌乃召頲詠之,立呈詩曰:「兔子死闌彈,持來掛竹竿。\CJKunderdot{試將明鏡照},\sidenote{「照」原作「召」,今據《唐代》本、《八千卷》本、《古今説部》本改。{\pagenum -85-}}何異月中看。」瓌大驚奇,驟加禮敬。頲由是學問日新,文章蓋代。及上平内難,一夕間制詔絡繹,無非頲出,代稱「小許公」也。

\vspace{1.5em}

12 上封太山回,車駕次上黨。路之父老,負擔壺漿,遠近迎謁。上皆親加存問,受其獻饋,錫賚有差。父老有先與上相識者,上悉賜酒食,與之話舊。故過村部,必令詢訪孤老喪疾之家,加吊恤之。父老忻忻然,莫不瞻戴叩乞駐留焉。\CJKunderdot{及車駕過金橋},\sidenote{原作「及車金橋」,今據《唐代》本、王讜《唐語林》卷四、《太平廣記》卷二一二補。}{\yuanshuzhu \CJKunderdot{橋在潞州},}\sidenote{原無此注,今據《唐代》本、王讜《唐語林》卷四、《太平廣記》卷二一二補。}御路縈轉,上見數十里間,旌纛鮮潔,羽衛整肅,顧謂左右曰:「張説言『勒兵三十萬,旌旗千里間。陝右上黨,至于太原。』(見《后土碑》)真才子也。」左右皆稱萬歲。上遂詔吴道玄、韋無忝、陳閎,令同製《金橋圖》。聖容及上所乘照夜白馬,陳閎主之;橋梁、山水、車輿、人物、草樹、\CJKunderdot{鹰鳥}、\sidenote{原作「雁鳥」,今據《唐代》本、《唐語林》、《太平廣記》改。}器仗、帷幕,吴道玄主之;狗馬、騾驢、牛羊、駱駞、\CJKunderdot{猫猴、猪𤝒四足之類},\sidenote{《唐語林》作「熊猿猪雞之類」。{\pagenum -86-}}韋無忝主之。圖成,時爲三絶焉。

\vspace{1.5em}

13 上幸蜀回,車駕次劍門,門左右巖壁峭絶。上謂侍臣曰:「劍門天險若此,自古及今,敗亡相繼,豈非在德不在險耶?」因駐蹕題詩曰:「\CJKunderdot{劍閣横空峻},\sidenote{《全唐詩》卷三唐玄宗《幸蜀回至劍門》:「劍閣横雲峻。」}鑾輿岀守回。翠屏千仞合,丹障五丁開。灌木縈旗轉,仙雲拂馬來。乘時方在德,嗟爾勒銘才。」其詩至德二年普安郡太守賈深勒于石壁,今存焉。

\vspace{1.5em}

14 賀知章祕書監,有高名,告老歸吴中,上嘉重之,\CJKunderdot{每别優異焉}。\sidenote{《太平廣記》卷二五五作「每事加異」。}知章將行,涕泣辭上。上曰:「何所欲?」知章曰:「臣有男未有定名,幸陛下賜之,歸爲鄉里榮。」上曰:「爲道之要莫若信,孚者,信也。履信思乎順,卿子必信順之人也,宜名之曰孚。」知章再拜而受命。知章久而謂人曰:「上何謔我耶?\CJKunderdot{我實吴人},\sidenote{原無「我」字,今據《太平廣記》卷二五五補。}孚乃瓜下爲子,豈非呼我爲瓜子耶?」{\pagenum -87-}

\vspace{1.5em}

15 上嘗坐朝,以手指上下按其腹。朝退,高力士進曰:「陛下向來數以手指按其腹,豈非聖體小不安耶?」上曰:「非也。吾昨夜夢遊月宫,諸仙娱予以上清之樂,寥亮清越,殆非人間所聞也。酣醉久之,合奏諸樂以送吾歸。其曲淒楚動人,杳杳在耳。吾回,以玉笛尋之,盡得之矣。坐朝之際,慮忽遺忘,故懷玉笛,時以手指上下尋,非不安。」力士再拜賀曰:「非常之事也。願陛下爲臣一奏之。」其聲寥寥然,不可名言也。力士又再拜,且請其名。上笑言曰:「此曲名《紫雲回》。」遂載于樂章。今太常刻石在焉。上封太山,進次滎陽旃然河上,見黑龍,\CJKunderdot{命弧矢親射之}。\sidenote{原無「親」字,今據王讜《唐語林》卷四、《太平廣記》卷四二〇補。}矢發,龍潛滅,自爾旃然伏流,于今百餘年矣。按旃然即濟水也,溢而爲滎,遂名旃然。《左傳》云:「\CJKunderdot{楚師次于旃然}。\sidenote{「次」原作「濟」,今據《古今説部》本、王讜《唐語林》卷四改。按《左傳·襄公》十八年:「遂涉潁,次于旃然。」}」是也。{\pagenum -88-}

\vspace{1.5em}

16 華岳雲臺觀中方之上,有山崛起半瓮之狀,名曰「瓮肚峰」。上賞望,嘉其高迥,欲於峰腹大鑿「開元」二字,填以白石,令百餘里望見。諫官上言,乃止。

\vspace{1.5em}

17 \CJKunderdot{上於弘農古函谷關得寶符},\sidenote(-31.5pt)[]{《太平廣記》卷一三六作「唐開元末於弘農古函谷關得寶符」。}白石篆文,正成「乘」字。識者解之云:「乘者,四十八,\CJKunderdot{所以示聖人御歷之數也,及帝幸蜀之來歲,正四十八}年,\sidenote(-52.5pt)[]{此二十一字原無,今據《唐代》本、《太平廣記》卷一三六補。}得寶之時,\CJKunderdot{天下歌之曰}:\sidenote(-26.25pt)[]{「歌之」原作「言之」,今據《唐代》本、《太平廣記》卷一三六改。}\CJKunderdot{得寶耶,弘農耶,弘農耶,得寶耶}!\sidenote(-15.75pt)[]{原作「得寶弘農得寶即」,不成文,今據《唐代》本、《太平廣記》卷一三六校補。}于今唱之得寶之年,\CJKunderdot{遂改元爲天寶也}。\sidenote(5.25pt)[]{原無「元爲」二字,今據《太平廣記》卷一三六補。}{\pagenum -89-}

\vspace{1.5em}

18 \CJKunderdot{上幸愛禄山爲子},\sidenote{《唐代》本、《太平廣記》卷二三八作「玄宗幸愛安禄山,呼禄山爲子」。王讜《唐語林》卷五作「上愛幸安禄山,呼之爲兒」。}嘗與貴妃於便殿同樂。禄山每就坐,\CJKunderdot{不拜上而拜妃}。\sidenote{《唐代》本、《太平廣記》卷二三八作「不幸玄宗而拜楊妃」。}上顧問:「此胡不拜我而拜妃子,意何在也?」禄山奏曰:「胡家即知有母,不知有父,故也。」上笑而捨之。祿山豐肥大腹,上嘗問曰:「\CJKunderdot{此胡腹中何物?其大如是}。」\sidenote{王讜《唐語林》卷五作「此腹中何物而大」,《太平廣記》卷二三八作「此胡腹中何物?其大乃爾」。按《新唐書·安禄山傳》:「胡腹中何有而大?」}禄山尋聲應曰:「腹中更無他物,唯赤心爾。」上以言誠而益親善之。{\pagenum -90-}

\vspace{1.5em}

19 一行將卒,留物一封,命弟子進於上。發而視之,乃蜀當歸也。上初不諭,及幸蜀回,乃知微旨,深歎異之。

\vspace{1.5em}

20 \CJKunderdot{羅公遠多祕術},\sidenote{「羅公遠」《太平廣記》卷七七作「羅思遠」,《新唐書·方伎傳》有《羅思遠》傳,叙事與此合。}最善隱形之法。上就公遠,雖傳受不肯盡其要。上每與同爲之,則隱没,人不能知。若自試,或餘衣帶,或露樸頭脚,\CJKunderdot{每被宫人知上所在}。\sidenote{《唐代》本、《太平廣記》卷七七作「宫中人每知帝所在也」。王讜《唐語林》卷五作「宫人每知上之所在也」。}\CJKunderdot{帝多方賜賚,或懼以死而求之,終不盡傳}。\sidenote{此十六字原無,今據《唐代》本、《太平廣記》卷七七補。王讜《唐語林》卷五作「百萬賜資,或臨之以死,公遠終不盡傳其術」。}上怒,命力士裹以油幞,置榨木下,\CJKunderdot{壓殺而埋棄之}。\sidenote{「埋棄之」《唐代》本、《太平廣記》卷七七作「埋瘞之」。}不旬日,有中使自蜀道回,逢公遠於路,乘騾而笑謂使者曰:「上之爲戲,一何虐耶?」{\pagenum -91-}

\vspace{1.5em}

21 萬廻師,閿鄉人也。神用若不足,謂愚而癡,無所知。雖父母亦以豚犬畜之。兄被戍役安西,音問隔絶。父母謂其誠死,日夕涕泣而憂思也。萬廻顧父感念甚,忽跪而言曰:「涕泣豈非憂兄耶?」\CJKunderdot{父母且信然},\sidenote{《太平廣記》卷九二作「父母且疑且信」。}萬廻曰:「詳思我兄所要者,衣裝、糗糧、\CJKunderdot{巾履之屬},\sidenote{原無「履」字,今據《太平廣記》卷九二補。}悉備之,\CJKunderdot{某將覲焉}。\sidenote{「覲焉」《小史》本、《説庫》本作「觀焉」。《太平廣記》卷九二作「往之」。}」忽一日朝賚所備,夕返其家,告父母曰:「兄平善矣。」發書視之,乃兄迹也,一家異之。弘農抵安西萬餘里,以其萬里而廻,故謂之萬廻也。居常貌{\pagenum -92-}如愚癡,忽有先覺異見,驚人神異也。上在藩邸,或遊行人間,萬廻於聚落街衢高聲曰:「天子來!」或曰:「聖人來!」其處信宿間,上必經過徘徊也。安樂公主,上之季妹也,附會韋氏,熱可炙手,道路懼焉。萬廻望其車騎,道唾曰:「血腥不可近也。」不旋踵而滅亡之禍及矣。上知萬廻非常人,内出二宫人,日夕侍奉,特敕於集賢院圖形焉。

\vspace{1.5em}

22 道士葉法善,精於符籙之術,上累拜爲鴻臚卿,\CJKunderdot{優禮待焉}。\sidenote{「待焉」《太平廣記》卷三六八作「特厚」。}法善居玄真觀,嘗有朝客數十人詣之,解帶淹留,滿座思酒。忽有人叩門,云:「麴秀才。」法善令人謂曰:「方有朝寮,未暇瞻晤,幸吾子異日見臨也。」語未畢,\CJKunderdot{有一美措傲睨直人},\sidenote{「美措」《説宛》本、《古今説部》本作「秀才」,《太平廣記》卷三六八作「措大」。}年二十餘,肥白可觀,笑揖諸公,居末席,伉聲談論,\CJKunderdot{援引古人},\sidenote{「古人」《太平廣記》卷三六八作「古今」。}一席不測,恐聳觀之。良久蹔起旋轉,法善謂諸公曰:「此子突入,語辯如此,豈非魃魅爲惑乎?\CJKunderdot{試與諸公避之}。\sidenote{《太平廣記》卷三六八作「試與諸公取劍備之」。}」號生復{\pagenum -93-}至,扼腕抵掌,論難鋒起,勢不可當。法善密以小劍擊之,\CJKunderdot{隨手失墜于階下},\sidenote{《太平廣記》卷三六八作「隨手喪元,墜于階下」。}化爲瓶植,一座驚懾,遽視其所,乃盈瓶醴醞也。咸大笑,飲之,其味甚嘉。坐客醉而揖其瓶曰:「麴生風味,不可忘也。」

\vspace{1.5em}

23 上命裴寬爲河南尹,寬性好釋氏,師事普寂襌師,旦夕造謁焉。居一日,寬詣寂,寂曰:「有少事,未暇款語,且請遲回休憩也。」寬乃屏賓從,止於空室。見寂絮滌正堂,焚香端坐。坐未久,忽聞扣門連聲云:「一行天師至。」一行入詣作禮,禮寂之足。禮訖,附耳密語,其貌絶恭。寂但顧云無不可者,語訖入禮,\CJKunderdot{禮訖又語,如是者三}。\sidenote{原作「禮語如是三」,有逸文,今據《太平廣記》卷九二補。}寂惟云:{\pagenum -94-}「是,是!」一行語訖,降堦入南堂,自闔其扉。寂乃徐命弟子云:「遣聲鍾,一行和尚滅度矣。」左右疾走視之,一如其言。後寂滅度,寬復衰絰。葬之日,徒步出城送之,甚爲搢紳所譏也。寬子諝復爲河南尹,\CJKunderdot{素好詼諧},\sidenote{「詼諧」原作「談諧」,今據《説宛》本、《古今説部》本改。}多異筆。嘗有投牒,誤書紙背。請判云:「者畔似那畔,那畔似者畔,我不可辭與你判,笑殺門前著靴漢。」又有婦人投狀争猫兒,狀云:「若是兒猫,即是兒猫;若不是兒猫,即不是兒猫。」諝大笑,判狀云:「猫兒不識主,旁我搦老鼠,兩家不須争,將來與裴諝。」遂納其猫兒,争者亦哂。

\vspace{1.5em}

24 安禄山初爲張韓公帳下走使之吏,\CJKunderdot{韓公常令祿山洗足}。\sidenote(-31.5pt)[]{原無「公」字,今據王讜《唐語林》卷三補。按張仁愿封韓國公,見新、舊《唐書》本傳,因韓非其姓,故當加公。}韓公脚下有黑點子,禄山因洗脚而竊窺之。韓公顧笑曰:「黑子,吾貴相也,獨汝窺之,亦能有之乎?」禄山曰:「某賤人也,不幸兩足皆有,\CJKunderdot{比將軍者黑而加文},\sidenote(-31.5pt)[]{「黑而加文」王讜《唐語林》卷三、吴曾《能改齋漫録》卷六作「色黑而加大」。}竟不知是何祥也。」韓公奇而觀之,益親厚之,約爲義兒,\CJKunderdot{而加薦寵焉}。\sidenote(-31.5pt)[]{《説宛》本、《古今説部》本作「爲加寵焉」。王讜《唐語林》卷三作「深加勉慰」。}{\pagenum -95-}

\vspace{1.5em}

25 \CJKunderdot{無畏三藏自天竺至},\sidenote(-31.5pt)[]{原無「至」字,今據《八千卷》本、《説庫》本、《小史》本、《太平廣記》卷九二補。}\CJKunderdot{所司引謁於玄宗},\sidenote{原作「上所可引謁」,不成語,今據《太平廣記》卷九二補、改。}上見而敬信焉。上謂三藏曰:「師自遠而來,困倦,欲於何方休息耶?」三藏進曰:「臣在天竺國時,聞西明寺宣律師持律第一,願依止焉。」上可之。宣律禁誡堅苦,焚修精潔。三藏飲酒食肉,言行麄易,往往乘醉而喧,穢污絪席。\CJKunderdot{宣律頗不甘心}。\sidenote{《太平廣記》卷九二作「宣律頗不能甘之」。}忽中夜,宣律捫虱,將投于地,三藏半醉,連聲呼曰:「律師撲死佛子!」宣律方知是神異人也,整衣作禮,投而師事之。宣律精苦之甚,常夜行道,臨堦墜墮,忽覺有人捧承其足。宣律顧視之,乃少年也。宣律遽問:「弟子何人,中夜在此?」少年曰:「某非常人,即毗沙王之子那吒太子也。護法之故,擁護和尚久矣。」宣律曰:「貧道修行,無事煩太子,\CJKunderdot{太子威神自在},\sidenote{原無「太子」二字,今據《説庫》本、《小史》本、《太平廣記》卷九二補。}西域有可作佛事者,願太子致之。」{\pagenum -96-}太子曰:「某有佛牙,賨事雖久,頭目猶捨,敢不奉獻。」\CJKunderdot{宣律求之},\sidenote{《太平廣記》卷九二作「宣律得之」,義長。}即今崇聖寺佛牙是也。

\vspace{1.5em}

26 太真妃最善於擊磬拊搏之音,泠泠然新聲,雖太常梨園之能人,莫加也。上令採藍田緑玉琢爲器,\CJKunderdot{上造簨簴流蘇之屬},\sidenote{「上造」《太平廣記》卷二〇四作「尚方造」,近是。}皆以金鈿珠翠珍怪之物雜飾之,又鑄二金師子,作拏攫騰奮之狀,各重二百餘斤,\CJKunderdot{以爲趺}\sidenote{原作「以扶」,今據《唐代》本、《太平廣記》卷二〇四改。},其他綵繪縟麗,制作神妙,一時無比也。上幸蜀回京師,樂器多忘失,獨玉磬偶在。上顧之悽然,不忍置於前,\CJKunderdot{促令送太常},\sidenote{「促令」《説宛》本、《古今説部》本作「遂命」。}\CJKunderdot{至今藏於太常正樂庫}。\sidenote{《太平廣記》卷二〇四作「至今藏於太樂署正聲庫者,是也」。}{\pagenum -97-}

\vspace{1.5em}

27 上所幸美人,忽夢人邀去,縱酒密會,任飲盡而歸,歸輒流汗,倦怠忽忽。後因從容盡白於上,上曰:「此必術人所爲也,汝若復往,但隨宜以物識之。」其夕熟寐,飄然又往。半醉,見石硯在前,乃密印手文於曲房屏風上,寤而具啓上。上乃潛以物色,令於諸宫觀求之。異日,於東明觀得其屏風,手文尚在,道士已遁矣。

\vspace{1.5em}

28 安西衙將劉文樹,口辯,善奏對,上每嘉之。文樹髭生頷下,貌類猿猴。上令黄幡綽嘲之。文樹切惡猿猴之號,乃密賂黄幡綽,祈不言之。\CJKunderdot{幡綽訊而進嘲}\sidenote{《太平廣記》卷二五五作「幡綽許而進嘲」。義長。}曰:「\CJKunderdot{可怜好箇劉文樹},\sidenote{原無「箇劉」二字,今據《太平廣記》卷二五五補。}髭鬚共頦頤别任。文樹面孔,不似瑚孫。\CJKunderdot{猢孫面孔},\sidenote{原無「面孔」二字,今據《太平廣記》卷二五五補。}强似文樹。」上知其賂{\pagenum -98-}遺,大笑之。

\vspace{1.5em}

29 平康坊南街廢蠻院,即李林甫舊宅也。林甫於正堂後别創一堂,製度彎曲,有却月之形,\CJKunderdot{名曰「偃月堂」}。\sidenote{原無「偃」字,今據《唐代》本、《太平廣記》卷三六二補。}木土秀麗精巧,當時莫儔也。林甫每欲破滅人家,\CJKunderdot{即入此堂精思極慮},\sidenote{「此堂」原作「月堂」,今據《太平廣記》卷三六二改。}喜悦而出,\CJKunderdot{其家必不存焉}。\sidenote{原無「其家」二字,今據《唐代》本、《太平廣記》卷三六二補。}及將敗,林甫於堂上見一物如人動,遍體被毛,毛如猪立,\CJKunderdot{鋸牙鈎爪長三尺餘},\sidenote{原無「長」字,今據《太平廣記》卷三六二補。}以擊林甫,目如電光而怒視之。林甫連叱不動,遂命弧矢。毛人笑而跳入前堂,堂中青衣,遇而暴卒;經于厩中,善馬皆死。\CJKunderdot{不累日而林甫卒}。\sidenote{《太平廣記》卷三六二作「不累月而林甫敗」。}{\pagenum -99-}

\vspace{1.5em}

30 太真妃常因妬媚,有語侵上,上怒甚,召高力士以轎軿送其家。妃悔恨號泣,抽刀剪髮授力士,曰:「珠玉珍異,皆上所賜,不足充獻,唯髮父母所生,可達妾意,望持此伸妾萬一慕戀之誠。」上得髮,揮涕憫然,遽命力士召歸。

\vspace{1.5em}

31 天寶初,上游華清宫,有劉朝霞者,\CJKunderdot{獻《駕幸温泉賦》},\sidenote{「駕」字原訛爲「賀」字,今據《太平廣記》卷二五〇改。}詞調倜儻,雜以俳諧,文多不載。今略其詞曰:若夫天寶二年,十月後兮臘月前,辦有司之供具,命駕幸于温泉。天門乾開,露神仙之輻湊;鑾輿劃出,驅甲仗以駢闐。青一隊兮黄一隊,熊踏胸兮豹拏背;{\pagenum -100-}朱一團兮綉一團,\CJKunderdot{玉鏤珂兮金鏤鞍}。\sidenote{「玉鏤珂」原作「玉鏤鈳」,今據《太平廣記》卷二五〇改。}《述德》云:直攫得盤古髓,搯得女蝸瓤,遮莫你古時千帝,豈如我今日三郎。其《自叙》云:别有窮奇蹭蹬,失路猖狂,\CJKunderdot{骨憧雖短},\sidenote{「骨憧」《八千卷》本作「骨懂」。}伎藝能長。夢裏幾回富貴,覺來依舊悽惶。今日是千年一遇,叩頭莫五角六張。帝覽而奇之,\CJKunderdot{將加賀},\sidenote{《太平廣記》卷二五〇作「將加殊賞」。}上命朝霞改去「五角六張」字。奏云:「臣草此賦時,有神助,自謂文不加點,筆不停綴,不願從天而改。」上顧曰:「真窮薄人也。」\CJKunderdot{授以春官衛上左焉}。\sidenote{《太平廣記》卷二五〇作「授以宫衛佐而止焉」。{\pagenum -101-}}

\newgeometry{
	%paperwidth=442.26pt,
	%paperheight=589.26pt,
	textwidth=315pt,
	textheight=442.26pt,
	bottom=63.63pt,
	right=63.63pt,
}%

\chapter{附\quad 録}

\section{晁公武郡齋讀書志卷二上}

《開元傳信記》一卷,右唐鄭綮記開元、天寶傳聞之事,故曰傳信。

\section{陳振孫直齋書録解題卷五}

《開元傳信記》一卷,唐吏部員外郎鄭綮撰,雜記開元、天寶時事。

\section{沈德壽抱經樓藏書志卷四八}

《開元傳信記》一卷,抄本,唐鄭綮撰。

\section{陸心源皕宋樓藏書志卷六四}

《開元傳信記》一卷,宋刊本,唐吏部員外郎鄭綮撰。自序。{\pagenum -102-}

\section{四庫全書總目卷一四二}

開天傳信記一卷,唐鄭綮撰。綮字蘊武,滎陽人。登進士第,累官右散騎常侍,好以詩謡託諷,昭宗意其有所蘊蓄,擢爲禮部侍郎同中書門下平章事,所謂「歇後鄭五作宰相,時事可知」者,即其人也。《舊唐書》本傳稱綮嘗歷監察、殿中,倉、户二員外,金、刑、右司三郎。而是書原本首署其官爲吏部員外郞,本傳顧未之及,或史文有所脱漏歟?書中皆記開元、天寶故事,凡三十二條。自序稱簿領之暇,搜求遺逸,期於必信,故以「傳信」爲名。其紀明皇戲游城南,王琚延過其家,謀誅韋氏一條,據《唐書·琚傳》,乃琚選補主簿,過謝太子,乘機進説,以除太平公主,並無先過琚家之事。司馬光作《通鑑》,亦不從是書,惟《新唐書》兼採之。然韋氏稱制時,琚方以王同皎黨亡命江都,安得復卜居韋杜?綮所記恐非事實,宜爲《通鑑》所不取。又如華陰見岳神、夢遊月宫、羅公遠隱形、葉法善符録諸事,亦語涉神怪,未能盡出雅馴。然行世既久,諸書言唐事者多沿用之,故録以備小説之一種焉。

\section{周中孚鄭堂讀書記卷六六}

《開天傳信記》一卷,百川學海本。唐鄭綮撰。(綮字蘊武,滎陽人,登進士第。昭宗時,{\pagenum -103-}官至禮部侍郎同中書門下平章事。)《四庫全書》著録,《新唐志》雜史類、《崇文目》雜史類、《讀書志》雜史類、《書録解題》雜史類、《通考》雜史類、《宋志》俱載之。是書乃其官郎時所自序,稱國朝故事,莫盛於開元、天寶之際,服膺簡策,管窺王業,參於閒聽,或有闕焉,承平之盛,不可殞墜,輒因簿領之暇,搜求遺逸,傳於必信,名曰《開天傳信記》。今觀其書,凡三十條,皆記開元、天寶間傳聞之事,間有失實。又如華陰岳神、夢遊月宫及盧公遠、葉法善諸事,尤不足以傳信。以文士久相引用,今亦不得而廢之焉。《説郛》、《歷代小史》、《學津討原》均收入之。

\section{余嘉錫四庫提要辨證卷一八}

嘉錫案:陳振孫《書録解題》卷五雜史類云:「《開天傳信記》一卷,唐吏部員外郎鄭綮撰,雜記開元、天寶時事。」勞格《唐郎官石柱題名考》卷四,據以補入「吏外」。{\pagenum -104-}

\vfill

\noindent 赤霓文字識別於壬寅年二月十二 排版於二月十四
\end{document}